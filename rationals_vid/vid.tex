\documentclass{beamer}
\usepackage{amsmath}
\usepackage{xcolor}
\usepackage{cancel}
\title{Operating With Rational Expressions}
\subtitle{Example Problems}

\begin{document}
	\frame {
		\titlepage
	}
	
	\frame {
		\frametitle{Problem 1: Simplification}
		Simplify the following expression as much as possible
		\begin{align*}
			\dfrac{x^2-16}{x^3 + 3x^2 - 16x - 48}
		\end{align*}
	}
	
	\frame {
		\frametitle{Problem 1: Simplification}
		Simplify the following expression as much as possible
		\begin{align*}
			\dfrac{x^2-16}{x^3 + 3x^2 - 16x - 48}\\	
			= 	\dfrac{\color{blue}(x - 4)(x + 4)}
			{x^3 + 3x^2 - 16x - 48}\\
		\end{align*}
	}

	\frame {
		\frametitle{Problem 1: Simplification}
		Simplify the following expression as much as possible
		\begin{align*}
			\dfrac{x^2-16}{x^3 + 3x^2 - 16x - 48}\\	
			= 	\dfrac{(x - 4)(x + 4)}
			{\color{blue}x^2(x + 3)\color{black} - 16x - 48}\\
		\end{align*}
	}

	\frame {
		\frametitle{Problem 1: Simplification}
		Simplify the following expression as much as possible
		\begin{align*}
			\dfrac{x^2-16}{x^3 + 3x^2 - 16x - 48}\\	
			= 	\dfrac{(x - 4)(x + 4)}
			{x^2(x + 3) - \color{blue} 16(x + 3)}\\
		\end{align*}
	}

	\frame {
		\frametitle{Problem 1: Simplification}
		Simplify the following expression as much as possible
		\begin{align*}
			\dfrac{x^2-16}{x^3 + 3x^2 - 16x - 48}\\	
			= 	\dfrac{(x - 4)(x + 4)}
			{\color{blue}(x + 3)(x^2 - 16)}\\
		\end{align*}
	}

	\frame {
		\frametitle{Problem 1: Simplification}
		Simplify the following expression as much as possible
		\begin{align*}
			\dfrac{x^2-16}{x^3 + 3x^2 - 16x - 48}\\	
			= 	\dfrac{(x - 4)(x + 4)}
			{(x + 3)\color{blue}(x - 4)(x + 4)}\\
		\end{align*}
	}

	\frame {
		\frametitle{Problem 1: Simplification}
		Simplify the following expression as much as possible
		\begin{align*}
			\dfrac{x^2-16}{x^3 + 3x^2 - 16x - 48}\\	
			= 	\dfrac{\color{blue}\cancel{\color{black}( x - 4)}\cancel{\color{black}(x + 4)}}
			{(x + 3)\color{blue}\cancel{\color{black} (x - 4)}\cancel{\color{black}(x + 4)}}\\
		\end{align*}
	}

	\frame {
		\frametitle{Problem 1: Simplification}
		Simplify the following expression as much as possible
		\begin{align*}
			\dfrac{x^2-16}{x^3 + 3x^2 - 16x - 48}\\	
			= 	\dfrac{\color{blue}\cancel{\color{black}( x - 4)}\cancel{\color{black}(x + 4)}}
			{(x + 3)\color{blue}\cancel{\color{black} (x - 4)}\cancel{\color{black}(x + 4)}}\\
			=	\color{green} \dfrac{1}{(x + 3)}
		\end{align*}
	}

	\frame {
		\frametitle{Problem 2: Addition}
		Write the following sum as a single fraction
		\begin{align*}
			\dfrac{2}{x + 3} + \dfrac{1}{2x^2 - 18}
		\end{align*}
	}

	\frame {
		\frametitle{Problem 2: Addition}
		Write the following sum as a single fraction
		\begin{align*}
			\dfrac{2}{x + 3} + \dfrac{1}{2x^2 - 18}\\
			=	\dfrac{2}{x + 3} + \dfrac{1}{\color{blue}2(x^2 - 9)}\\
		\end{align*}
	}

	\frame {
		\frametitle{Problem 2: Addition}
		Write the following sum as a single fraction
		\begin{align*}
			\dfrac{2}{x + 3} + \dfrac{1}{2x^2 - 18}\\
			=	\dfrac{2}{x + 3} + \dfrac{1}{2\color{blue}(x - 3)(x + 3)}\\
		\end{align*}
	}

	\frame {
		\frametitle{Problem 2: Addition}
		Write the following sum as a single fraction
		\begin{align*}
			\dfrac{2}{x + 3} + \dfrac{1}{2x^2 - 18}\\
			=	\dfrac{2}{x + 3} + \dfrac{1}{2(x - 3)(x + 3)}\\
			=	\dfrac{2 \color{blue}\cdot 2 \cdot (x - 3)}{\color{black} (x + 3) \color{blue} \cdot 2 \cdot (x - 3)} + \dfrac{1}{2(x - 3)(x + 3)}
		\end{align*}
	}

	\frame {
		\frametitle{Problem 2: Addition}
		Write the following sum as a single fraction
		\begin{align*}
			\dfrac{2}{x + 3} + \dfrac{1}{2x^2 - 18}\\
			=	\dfrac{2}{x + 3} + \dfrac{1}{2(x - 3)(x + 3)}\\
			=	\dfrac{\color{blue}4x - 12}{\color{black} 2(x + 3)(x - 3)} + \dfrac{1}{2(x - 3)(x + 3)}
		\end{align*}
	}

	\frame {
		\frametitle{Problem 2: Addition}
		Write the following sum as a single fraction
		\begin{align*}
			\dfrac{2}{x + 3} + \dfrac{1}{2x^2 - 18}\\
			=	\dfrac{2}{x + 3} + \dfrac{1}{2(x - 3)(x + 3)}\\
			=	\dfrac{4x - 12 \color{blue} + 1}{\color{black} 2(x + 3)(x - 3)}
		\end{align*}
	}

	\frame {
		\frametitle{Problem 2: Addition}
		Write the following sum as a single fraction
		\begin{align*}
			\dfrac{2}{x + 3} + \dfrac{1}{2x^2 - 18}\\
			=	\dfrac{2}{x + 3} + \dfrac{1}{2(x - 3)(x + 3)}\\
			=	\dfrac{4x\color{blue} - 11}{2\color{blue}(x^2 - 9)}
		\end{align*}
	}

	\frame {
		\frametitle{Problem 2: Addition}
		Write the following sum as a single fraction
		\begin{align*}
			\dfrac{2}{x + 3} + \dfrac{1}{2x^2 - 18}\\
			=	\dfrac{2}{x + 3} + \dfrac{1}{2(x - 3)(x + 3)}\\
			=	\dfrac{4x - 11}{2(x^2 - 9)}\\
			=	\color{green}\dfrac{4x - 11}{2x^2 - 18}
		\end{align*}
	}

	\frame {
		\frametitle{Problem 3: Multiplication}
		Multiply and simply as much as possible
		\begin{align*}
			\dfrac{12x^2 - 156x + 432}{6x^2 - 64x + 270} \cdot \dfrac{4x^2 - 32x + 60}{8x^2 - 8x - 96}
		\end{align*}
	}

	\frame {
		\frametitle{Problem 3: Multiplication}
		Multiply and simply as much as possible
		\begin{align*}
			\dfrac{12x^2 - 156x + 432}{6x^2 - 64x + 270} \cdot \dfrac{4x^2 - 32x + 60}{8x^2 - 8x - 96}\\
			=	\dfrac{\color{blue}12 \color{black} (x^2 - 13x + 36)}{\color{blue}6\color{black}(x^2 - 14x + 45)} \cdot \dfrac{\color{blue}4\color{black}(x^2 - 8x + 15)}{\color{blue}8\color{black}(x^2 - x - 12)}
		\end{align*}
	}

	\frame {
		\frametitle{Problem 3: Multiplication}
		Multiply and simply as much as possible
		\begin{align*}
			\dfrac{12x^2 - 156x + 432}{6x^2 - 64x + 270} \cdot \dfrac{4x^2 - 32x + 60}{8x^2 - 8x - 96}\\
			=	\dfrac{\color{blue}\cancel{\color{black}12} \color{black} (x^2 - 13x + 36)}{\color{blue}\cancel{\color{black}6}\color{black}(x^2 - 14x + 45)} \cdot \dfrac{\color{blue}\cancel{\color{black}4}\color{black}(x^2 - 8x + 15)}{\color{blue}\cancel{\color{black}8}\color{black}(x^2 - x - 12)}
		\end{align*}
	}

	\frame {
		\frametitle{Problem 3: Multiplication}
		Multiply and simply as much as possible
		\begin{align*}
			\dfrac{12x^2 - 156x + 432}{6x^2 - 64x + 270} \cdot \dfrac{4x^2 - 32x + 60}{8x^2 - 8x - 96}\\
			=	\dfrac{(x^2 - 13x + 36)}{(x^2 - 14x + 45)} \cdot \dfrac{(x^2 - 8x + 15)}{(x^2 - x - 12)}\\
			=	\color{blue}\dfrac{(x - 4)(x - 9)}{(x - 5)(x - 9)} \cdot \dfrac{(x - 5)(x - 3)}{(x - 4)(x + 3)}
		\end{align*}
	}

	\frame {
		\frametitle{Problem 3: Multiplication}
		Multiply and simply as much as possible
		\begin{align*}
			\dfrac{12x^2 - 156x + 432}{6x^2 - 64x + 270} \cdot \dfrac{4x^2 - 32x + 60}{8x^2 - 8x - 96}\\
			=	\dfrac{(x^2 - 13x + 36)}{(x^2 - 14x + 45)} \cdot \dfrac{(x^2 - 8x + 15)}{(x^2 - x - 12)}\\
			=	\dfrac{\color{blue}\cancel{\color{black}(x - 4)}\color{blue}\cancel{\color{black}(x - 9)}}{\color{blue}\cancel{\color{black}(x - 5)}\color{blue}\cancel{\color{black}(x - 9)}} \cdot \dfrac{\color{blue}\cancel{\color{black}(x - 5)}\color{black}(x - 3)}{\color{blue}\cancel{\color{black}(x - 4)}\color{black}(x + 3)}
		\end{align*}
	}

	\frame {
		\frametitle{Problem 3: Multiplication}
		Multiply and simply as much as possible
		\begin{align*}
			\dfrac{12x^2 - 156x + 432}{6x^2 - 64x + 270} \cdot \dfrac{4x^2 - 32x + 60}{8x^2 - 8x - 96}\\
			=	\dfrac{(x^2 - 13x + 36)}{(x^2 - 14x + 45)} \cdot \dfrac{(x^2 - 8x + 15)}{(x^2 - x - 12)}\\
			=	\dfrac{\color{blue}\cancel{\color{black}(x - 4)}\color{blue}\cancel{\color{black}(x - 9)}}{\color{blue}\cancel{\color{black}(x - 5)}\color{blue}\cancel{\color{black}(x - 9)}} \cdot \dfrac{\color{blue}\cancel{\color{black}(x - 5)}\color{black}(x - 3)}{\color{blue}\cancel{\color{black}(x - 4)}\color{black}(x + 3)}\\
			=	\color{green}\dfrac{x - 3}{x + 3}
		\end{align*}
	}

\end{document}
