\documentclass{beamer}
\usepackage{amsmath}
\usepackage{xcolor}
\usepackage{cancel}
\title{Polynomial Division}
\subtitle{Example Problems}

\begin{document}
	\frame {
		\titlepage
	}
	
	\frame {
		\frametitle{Problem 1: Dividing Polynomials by x with Remainders}
		Divide the polynomial by x and express your answer in the form p(x) + k/x where p is a polynomial and k is an integer
		\begin{align*}
		\dfrac{x^4+x^2+5}{x}\\
		\end{align*}
	}
	
	\frame {
		\frametitle{Problem 1: Dividing Polynomials by x with Remainders}
		Divide the polynomial by x and express your answer in the form p(x) + k/x where p is a polynomial and k is an integer
		\begin{align*}
		\dfrac{x^4+x^2+5}{x}\\
		=\\
		\color{blue}\dfrac{x^4}{x} + \dfrac{x^2}{x} + \dfrac{5}{x}\\
		\end{align*}
	}

	\frame {
		\frametitle{Problem 1: Dividing Polynomials by x with Remainders}
		Divide the polynomial by x and express your answer in the form p(x) + k/x where p is a polynomial and k is an integer
		\begin{align*}
		\dfrac{x^4}{x} + \dfrac{x^2}{x} + \dfrac{5}{x}\\
		=\\
		\color{blue} x^3+x+\color{black}\dfrac{x}{5}\\
		\end{align*}
	}

	
	\frame {
		\frametitle{Problem 2: Dividing Quadratics by Linear Expressions with Remainders}
		Divide the polynomial by the linear expression and express your answer in the form 
		\begin{align*}
		\sqrt{\color{blue}16}\\ 
		=\\
		\color{blue}4\\
		\end{align*}
	}

	\frame {
		\frametitle{Problem 2: Evaluating Radical Expressions}
		Simplify the radical expression.
		\begin{align*}
			(\dfrac{1}{4})^{-1/4} \times (64)^{-1/4}\\
		\end{align*}
	}

	\frame {
		\frametitle{Problem 2: Evaluating Radical Expressions}
		Simplify the radical expression.
		\begin{align*}
			(\dfrac{1}{4})^{-1/4} \times (64)^{-1/4}\\
			=\\
			(\color{blue}\dfrac{64}{4})\color{black}^{-1/4}\\
		\end{align*}
	}

	\frame {
		\frametitle{Problem 2: Evaluating Radical Expressions}
		Simplify the radical expression.
		\begin{align*}
			(16)^{-1/4}\\
			=\\
			\color{blue}\dfrac{1}{(16)^{1/4}}\\
		\end{align*}
	}

	\frame {
		\frametitle{Problem 2: Evaluating Radical Expressions}
		Simplify the radical expression.
		\begin{align*}
			\dfrac{1}{(16)^{1/4}}\\
			=\\
			\color{blue}\dfrac{1}{2}\\
		\end{align*}
	}

	\frame {
		\frametitle{Problem 3: Evaluating Radical Expressions}
		Simplify the radical expression.
		\begin{align*}
			\dfrac{(4)^{1/5}}{\sqrt[5]{128}}\\
		\end{align*}
	}

	\frame {
		\frametitle{Problem 3: Evaluating Radical Expressions}
		Simplify the radical expression.
		\begin{align*}
			\dfrac{(4)^{1/5}}{\sqrt[5]{128}}\\
			=\\
			\color{blue}\sqrt[5]{\dfrac{4}{128}}\\
		\end{align*}
	}

	\frame {
		\frametitle{Problem 3: Evaluating Radical Expressions}
		Simplify the radical expression.
		\begin{align*}
			\sqrt[5]{\dfrac{4}{128}}\\
			=\\
			\color{blue}\sqrt[5]{\dfrac{1}{32}}\\
		\end{align*}
	}

	\frame {
		\frametitle{Problem 3:Evaluating Radical Expressions}
		Simplify the radical expression.
		\begin{align*}
		\sqrt[5]{\dfrac{1}{32}}\\
		=\\
		\color{blue}\dfrac{1}{\sqrt[5]{32}}\\
		\end{align*}
	}
	
	\frame{
		\frametitle{Problem 3: Evaluating Radical Expressions}
		Simplify the radical expression.
		\begin{align*}
		\dfrac{1}{\sqrt[5]{32}}\\
		=\\
		\color{blue}\dfrac{1}{2}\\
		\end{align*}
	}
	
	\frame{
		\frametitle{Congrats!}
		I hope you learned something and enjoyed this video!
	}
\end{document}
