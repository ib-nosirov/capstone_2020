\documentclass[leqno, 11pt]{article}
\usepackage{multicol}
\usepackage{amsmath}
\usepackage{xcolor}
\begin{document}
\title{Logarithmic Equations and Inequalities}

Solve the following logarithmic equations.\\

\begin{multicols}{2}
	\begin{align*}
    log_2(x - 1) + log_2(x + 2) = 2\\
	  \color{blue}
	  	log_2(x - 1)(x + 2) = 2\\
	  \color{blue}
		(x - 1)(x + 2) = 4\\
	  \color{blue}
		x^2 + x - 6 = 0\\
	  \color{blue}
		(x + 3)(x - 2) = 0\\
	  \color{blue}
		x = -3, 2\\ 
	  \color{blue}
		\text{Since x \textgreater 0},\\
	  \color{blue}
		x - 1 > 0\text{, or } x + 2 > 0\\
	  \color{blue}
		\text{which results to } x > 1
	\end{align*}
  \begin{equation}
    2log(x - 1) = log(x + 1)
  \end{equation}
\end{multicols}

\vspace{\stretch{1}}

\begin{multicols}{2}
  \begin{equation}
	log_5(x + 1) + log_5(x - 3) = 1
  \end{equation}\break
  \begin{equation}
	log_2(x + 1) = log_2(2 - x) + 1
  \end{equation}
\end{multicols}

\vspace{\stretch{1}}

\clearpage

\begin{multicols}{2}
  \begin{equation}
	  (1 + log_2x) \cdot log_2x = 2
  \end{equation}\break
  \begin{equation}
	  (log_3x)^2 - 5log_3x + 6 = 0
  \end{equation}
\end{multicols}

\vspace{\stretch{1}}

\begin{multicols}{2}
  \begin{equation}
	  (log_2x)^2 = log_2x^2 + 3
  \end{equation}\break
  \begin{equation}
	  2log_2x - 3log_x2 + 5 = 0
  \end{equation}
\end{multicols}

\vspace{\stretch{1}}

\clearpage

\end{document}
