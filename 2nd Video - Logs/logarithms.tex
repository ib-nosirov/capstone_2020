\documentclass{beamer}
\usepackage{amsmath}
\usepackage{xcolor}
\usepackage{cancel}
\title{More Logarithms}
\subtitle{Example Problems}

\begin{document}
	\frame {
		\titlepage
	}
	
	\frame {
		\frametitle{Problem 1: Expanding Logarithms}
		Expand the logarithm fully using the properties of logs. Express the final answer in terms of 
log x and log y.
		\begin{align*}
			log \dfrac{4x^2}{y^5}\\
		\end{align*}
	}
	
	\frame {
		\frametitle{Problem 1: Expanding Logarithms}
		Expand the logarithm fully using the properties of logs. Express the final answer in terms of 
log x and log y.
		\begin{align*}
			log \dfrac{4x^2}{y^5}\\
			=\\
		 	\color{blue}log4 +logx^2 - logy^5\\
		\end{align*}
	}

	\frame {
		\frametitle{Problem 1: Expanding Logarithms}
		Expand the logarithm fully using the properties of logs. Express the final answer in terms of 
log x and log y.
		\begin{align*}
			log4 + logx^2 - log y^5\\
			=\\
			log4 + \color{blue}2logx \color{black} - \color{blue}5logy\\
		\end{align*}
	}

	
	\frame {
		\frametitle{Problem 2: Expanding Logarithms}
		Find the numerical value of the log expression.
		\begin{align*}
			log a = 3\\
			log b = 4\\
			log c = 5\\
			log c^4b^5 \sqrt[4]{a^5} 
		\end{align*}
	}

	\frame {
		\frametitle{Problem 2: Expanding Logarithms}
		find the numerical value of the log expression.
		\begin{align*}
			log c^4b^5 \sqrt[4]{a^5}\\
		  =\\
		  \color{blue}logc^4 + logb^5 + loga^{5/4}\\
		\end{align*}
	}

	\frame {
		\frametitle{Problem 2: Expanding Logarithms}
		Find the numerical value of the log expression.
		\begin{align*}
			logc^4 + logb^5 + loga^{5/4}\\
		=\\
		\color{blue}4 \color{black}logc + \color{blue}5 \color{black}logb + \color{blue}\dfrac{5}{4} \color{black}loga\\
		\end{align*}
	}

	\frame {
		\frametitle{Problem 2: Expanding Logarithms}
		Find the numerical value of the log expression.
		\begin{align*}
			4logc + 5logb + \dfrac{5}{4}loga\\
			=\\
			4\color{blue}(5) \color{black} + 5 \color{blue}(4) \color{black} + \dfrac{5}{4} \color{blue}(3)\\
		\end{align*}
	}

	\frame {
		\frametitle{Problem 2: Expanding Logarithms}
		Find the numerical value of the log expression.
		\begin{align*}
			20 + 20 + \dfrac{15}{4}\\
			=\\
			\color{blue} \dfrac{175}{4}
		\end{align*}
	}

	\frame {
		\frametitle{Problem 3: Logarithmic Form}
		Solve for the positive solution of x.
		\begin{align*}
			log_x\dfrac{1}{5} = \dfrac{1}{4}\\
		\end{align*}
	}

	\frame {
		\frametitle{Problem 3: Logarithmic Form}
		Solve for the positive solution of x.
		\begin{align*}
			log_x\dfrac{1}{5} = \dfrac{1}{4}\\
			=\\
			x\color{blue}^{1/4} \color{black} = \color{blue} \dfrac{1}{5}\\
		\end{align*}
	}

	\frame {
		\frametitle{Problem 3: Logarithmic Form}
		Solve for the positive solution of x.
		\begin{align*}
			x^{1/4} = \dfrac{1}{5}\\
			=\\
			x^{1/4^{\color{blue}-4}} \color{black} = \dfrac{1}{5}^{\color{blue}-4}\\
		\end{align*}
	}

	\frame {
		\frametitle{Problem 3: Logarithmic Form}
		Solve for the positive solution of x.
		\begin{align*}
		x^{1/4^{-4}} = \dfrac{1}{5}^{-4}\\
		=\\
		\color{blue}x \color{black}= \color{blue}5^{4}\\
		\end{align*}
	}
	
	\frame{
		\frametitle{Problem 3: Logarithmic Form}
		Solve for the positive solution of x.
		\begin{align*}
		x = 5^{4}\\
		=\\
		\color{blue}625\\
		\end{align*}
	}
	
	\frame{
		\frametitle{Congrats!}
		I hope you learned something and enjoyed this video!
	}
\end{document}

